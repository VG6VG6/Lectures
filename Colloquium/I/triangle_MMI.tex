\subsection{Неравенство треугольника для действительных и комплексных чисел, геометрическое и алгебраическое доказательства.}

\subsubsection{Неравенство треугольника}
\begin{figure}[h]
    \centering
    \includegraphics[width=0.5\linewidth]{images/Неравенство(вектора).png}
    \caption{Геометрический смысл неравенства треугольника: длина стороны \( |\vec{a} + \vec{b}| \) не превосходит суммы длин сторон \( |\vec{a}| + |\vec{b}| \).}
    \label{fig:triangle}
\end{figure}
\begin{tcolorbox}
\textbf{Теорема (Неравенство треугольника):}
Для любых комплексных чисел $z_1, z_2$ справедливо:
$$|z_1 + z_2| \leq |z_1| + |z_2|$$
\end{tcolorbox}

\begin{tcolorbox}[title=Доказательство, breakable]
$|a+b| \leq |a| + |b|$
\begin{enumerate}
    \item $a \geq 0\ (|a| \geq |b|)$ \\
    $a + b \geq 0$, то $|a+b| = a+b$.\\
    $a + b \leq 0$, то $|a+b| = -(a+b) = -a-b \leq |a|+|b|$
    \item $|a-b| \geq ||a|-|b||$\\
    $a = (a - b) + b$ по н.т.: $|a + 0| \leq |a-b|+|b| \Rightarrow |a-b| \geq |a|-|b|$ \\
    Аналогично $|b| \leq |b-a|+|a| \Rightarrow |a-b| \geq |b|-|a|$ \\
    Получим, что 
    $
    \begin{cases}
        |a-b| \geq |a|-|b|\\
        |a-b| \geq |-(|a|-|b|)
    \end{cases}
    \Rightarrow |a-b| \geq ||a|-|b||
    $
\end{enumerate}
\end{tcolorbox}

\begin{tcolorbox}[title=Следствие, breakable]
$$||z_1| - |z_2|| \leq |z_1 \pm z_2| \leq |z_1| + |z_2|$$
\end{tcolorbox}
\subsection{-Формулы Моргана}
\subsection{Метод математической индукции (ММИ). Прямая индукция. Формула Бинома Ньютона}

\subsubsection{Метод математической индукции (ММИ)}
\begin{tcolorbox}
\textbf{Алгоритм доказательства по индукции:}
\begin{enumerate}
    \item \textbf{База индукции:} Проверить утверждение для $n = 1$.
    \item \textbf{Индукционное предположение:} Предположить, что утверждение верно для $n = k$.
    \item \textbf{Индукционный переход:} Доказать, что из этого следует верность утверждения для $n = k+1$ (Прямая индукция).
\end{enumerate}
\end{tcolorbox}

\subsubsection{Бином Ньютона}
\begin{tcolorbox}
\textbf{Определение.}\\
\textit{Биномиальный коэффициент}:
$C_n^k = \dfrac{n!}{k!(n-k)!}$, где $n,k \in \mathbb{N}_0$\\
\end{tcolorbox}

\begin{tcolorbox}
\textbf{Формула бинома Ньютона:}
$$(a+b)^n = \sum_{k=0}^{n} C_n^k a^{n-k}b^k$$
\end{tcolorbox}

\begin{tcolorbox}[title=Доказательство по ММИ, breakable]
\textbf{База индукции:} Для $n=1$:
$$(a+b)^1 = a + b$$
$$\sum_{k=0}^{1} C_1^k a^{1-k}b^k = C_1^0 a^1 b^0 + C_1^1 a^0 b^1 = 1 \cdot a \cdot 1 + 1 \cdot 1 \cdot b = a + b$$
База индукции доказана.

\textbf{Индукционное предположение:} Предположим, формула верна для $n = m$:
$$(a+b)^m = \sum_{k=0}^{m} C_m^k a^{m-k}b^k$$

\textbf{Индукционный переход:} Докажем для $n = m+1$.
Рассмотрим левую часть:
$$(a+b)^{m+1} = (a+b) \cdot \sum_{k=0}^{m} C_m^k a^{m-k}b^k$$
Раскроем скобки:
$$= \sum_{k=0}^{m} C_m^k a^{m+1-k}b^k + \sum_{k=0}^{m} C_m^k a^{m-k}b^{k+1}$$
Во второй сумме сделаем замену индекса $j = k+1$:
$$= \sum_{k=0}^{m} C_m^k a^{m+1-k}b^k + \sum_{j=1}^{m+1} C_m^{j-1} a^{m+1-j}b^{j}$$
Теперь объединим суммы, выделяя крайние слагаемые:
$$= C_m^0 a^{m+1} + \sum_{k=1}^{m} \left[ C_m^k + C_m^{k-1} \right] a^{(m+1)-k}b^k + C_m^m b^{m+1}$$
Используем свойство биномиальных коэффициентов:
$$C_m^k + C_m^{k-1} = C_{m+1}^k$$
Учитывая, что $C_m^0 = C_{m+1}^0 = 1$ и $C_m^m = C_{m+1}^{m+1} = 1$, получаем:
$$(a+b)^{m+1} = \sum_{k=0}^{m+1} C_{m+1}^k a^{(m+1)-k}b^k$$
Индукционный переход завершён.
\end{tcolorbox}

\subsection{ММИ. Обратная индукция. Неравенство между средним арифметическим и средним геометрическим.}

\subsubsection*{Неравенство о средних}
\begin{tcolorbox}
\textbf{Теорема (Неравенство между средним арифметическим и средним геометрическим):}
Для любых $a_1, a_2, \dots, a_n \geq 0$ справедливо:
$$\frac{a_1 + a_2 + \dots + a_n}{n} \geq \sqrt[n]{a_1 a_2 \dots a_n}$$
Равенство достигается тогда и только тогда, когда $a_1 = a_2 = \dots = a_n$.
\end{tcolorbox}

\begin{tcolorbox}[title=Доказательство по ММИ (метод Коши / метод обратой индукции), breakable]
Докажем теорему в три этапа.

\textbf{1. База индукции для степеней двойки ($n=2^m$).}
\begin{itemize}
    \item \textbf{Для $n=2$:} Докажем $\frac{a_1 + a_2}{2} \geq \sqrt{a_1 a_2}$.
    $$(a_1 - a_2)^2 \geq 0 \Rightarrow a_1^2 - 2a_1a_2 + a_2^2 \geq 0 \Rightarrow a_1^2 + 2a_1a_2 + a_2^2 \geq 4a_1a_2 \Rightarrow$$
    $$\Rightarrow (a_1 + a_2)^2 \geq 4a_1a_2 \Rightarrow \frac{a_1 + a_2}{2} \geq \sqrt{a_1 a_2}$$
    \item \textbf{Предположим, неравенство верно для $n = k$.}
    \item \textbf{Докажем для $n = 2k$:}
    $$\frac{a_1 + \dots + a_{2k}}{2k} = \frac{\frac{a_1 + \dots + a_k}{k} + \frac{a_{k+1} + \dots + a_{2k}}{k}}{2} \geq$$
    $$\geq \frac{\sqrt[k]{a_1 \dots a_k} + \sqrt[k]{a_{k+1} \dots a_{2k}}}{2} \geq \sqrt{\sqrt[k]{a_1 \dots a_k} \cdot \sqrt[k]{a_{k+1} \dots a_{2k}}} = \sqrt[2k]{a_1 \dots a_{2k}}$$
\end{itemize}

\textbf{2. Докажем, что если неравенство верно для $n$, то оно верно и для $n-1$.}
Рассмотрим $a_1, a_2, \dots, a_{n-1} \geq 0$. Пусть
$$a_n = \frac{a_1 + a_2 + \dots + a_{n-1}}{n-1}$$
Для набора из $n$ чисел неравенство верно:
$$\frac{a_1 + \dots + a_{n-1} + a_n}{n} \geq \sqrt[n]{a_1 a_2 \dots a_{n-1} a_n}$$
Подставим $a_n$:
$$\frac{(a_1 + \dots + a_{n-1}) + \frac{a_1 + \dots + a_{n-1}}{n-1}}{n} = \frac{a_1 + \dots + a_{n-1}}{n-1} = a_n$$
Таким образом:
$$a_n \geq \sqrt[n]{a_1 a_2 \dots a_{n-1} a_n}$$
Возведём в степень $n$:
$$a_n^n \geq a_1 a_2 \dots a_{n-1} a_n \Rightarrow a_n^{n-1} \geq a_1 a_2 \dots a_{n-1}$$
Извлекая корень $(n-1)$-й степени:
$$a_n \geq \sqrt[n-1]{a_1 a_2 \dots a_{n-1}} \Rightarrow \frac{a_1 + \dots + a_{n-1}}{n-1} \geq \sqrt[n-1]{a_1 a_2 \dots a_{n-1}}$$

\textbf{3. Завершение доказательства.}
Мы доказали, что:
\begin{enumerate}
    \item Неравенство верно для $n=2$ (а значит, для $n=4,8,16,\dots$)
    \item Из верности для $n$ следует верность для $n-1$
\end{enumerate}
Следовательно, неравенство верно для любого натурального $n$.
\end{tcolorbox}
