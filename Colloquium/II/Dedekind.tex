\subsection{Дедекиндовы сечения. Определение действительных чисел по Дедекинду. Полнота $\R$ по Дедекинду. (Полноты я не нашёл)}

\subsubsection{Неполнота рациональных чисел.}
$r = \cfrac{p}{q},p\in \Z,q\in\N $\\
Дробь можно сделать несократимой.\\
Пусть $(\cfrac{p}{q})^2 = 2$ - несократимая дробь $\Rightarrow p^2 = 2q^2 \Rightarrow p = 2k$ (p - чётное число) \\
$k^2 = 2q^2 \Rightarrow q^2 = 2k^2 \Rightarrow $ q - чётное число
\\т.к. дробь несократимая, а числитель и знаменатель чётные, то она на самом деле сократимая. Противоречие! \\
Значит это число \textbf{нельзя} представить рациональной дробью.
\begin{figure}[h]
    \includegraphics[width=0.2\linewidth]{images/Дедекиндовы сечения/Начало дедекиндовых сечений.png}
\end{figure}\\
Если на оси отметить все рациональные числа точками, то $\sqrt{2}$ - будет выколотой точкой.
\subsubsection*{Отрезки}
$I_n = [a_n, b_n] = \{ r \in \Q\; |\; a_n \leq r \leq b_n \}$, где $a_n, b_n \in \Q$
\begin{tcolorbox}
Будем считать, что $\forall n [a_{n+1}, b_{n+1}] \subset [a_n, b_n]$ - \textbf{вложенные отрезки.}\\
Это значит, что $a_n \leq a_{n+1} \leq b_{n+1} \leq b_n$ и следующий отрезок меньше предыдущего.
\end{tcolorbox}
\begin{tcolorbox}
    $b_n - a_n \xrightarrow{n\to\infty} 0$ - \textbf{стягивающиеся отрезки.}
\end{tcolorbox}
\textbf{Дальше будем подразумеваться}, что все последовательности $\{I_n\}$ - вложенные и стягивающиеся в точку.
\begin{figure}[h]
    \includegraphics[width=0.25\linewidth]{images/Дедекиндовы сечения/Смысл отрезков.png}
\end{figure}\\
Что нам дают такие отрезки: $\forall r$ \\
\begin{equation}
    \exists n: r < a_n \Rightarrow \forall m > n:\; r < a_m
\end{equation}
\begin{equation}
    \exists n: r > b_n \Rightarrow \forall m > n:\; r > b_m
\end{equation}
\begin{equation}
    \forall n: a_n \leq r \leq b_n
\end{equation}
\begin{enumerate}
    \item левый класс для $\{I_n\}$ (всегда не пуст)
    \item правый класс для $\{I_n\}$ (всегда не пуст)
    \item центральный класс для $\{I_n\}$ (может быть пустым учитывая $r \in \Q)$. Не может содержать более 1 Q числа.
\end{enumerate}
Слово "класс" подразумевает множество.
\begin{tcolorbox}
    \textbf{Дедекиндово сечение}  - множество рациональных чисел (Q) порождённое последовательностью $\{I_n\}$.
\end{tcolorbox}
Тогда каждому действительному числу будет соответствовать своё дедекиндово сечение.

\begin{tcolorbox}
\textbf{Определение}\\
\textit{$\{I_n\}$ и $\{I_n``\}$ - \textbf{эквиваленты}, если они порождают одинаковые разбиения на классы}
\end{tcolorbox}

\begin{tcolorbox}
\textbf{Теорема 1}\\
$\{ I_n \}\sim$ эквивалентна $\{ I_n` \} \Leftrightarrow$
\begin{enumerate}
    \item $\forall n \;\; a_n - a_n` \xrightarrow{n \to \infty} 0 $\\
    ИЛИ
    \item $\forall n a_n \leq b_n`,\; a_n` \leq b_n$
\end{enumerate}
\end{tcolorbox}

\begin{tcolorbox}[title=Доказательство Т1, breakable]
\includegraphics[width=0.4\linewidth]{images/Дедекиндовы сечения/Доказательство Т1.png}\\
\textbf{Только п.1.} \\
$\Rightarrow$: $\{ I_n \} \sim \{ I_n` \} \Rightarrow a_n-a_n` \rightarrow 0$\\
от противного: тогда $a_n-a_n` \nrightarrow 0 \Leftrightarrow \\ 
\exists \eps>0:\; \forall N\;\; \exists n > N\; |a_n - a_n`| \geq \eps $ \\
$\Rightarrow $ для бесконечно многих номеров либо $a_n-a_n` > \eps$, либо $a_n` - a_n > \eps$\\
Пусть для бесконечно многих номеров $a_n - a_n` > \eps$ \\
$\exists n_\eps$ длина $[a_n`, b_n`] < \cfrac{\eps}{2}$\\
$b_n`-a_n` \rightarrow 0 \; \exists n_\eps: \; \forall n > n_\eps\; |b_n`-a_n`| < \cfrac{\eps}{2}$\\
Т.к. $\exists r \in [b_n`, a_n]$, то она принадлежит правому классу $\{I_n`\}$ и левому классу $\{I_n\}$. Последовательности не эквивалентны.\\

$\Leftarrow: a_n - a_n`\; \xrightarrow{n\rightarrow \infty} 0 \Rightarrow \{ I_n \} \sim \{ I_n` \}$\\
От противного: пусть $\{ I_n \} \nsim \{ I_n` \}$, то есть $\exists r \in Q\; r$ из левого класса для одной и центрального или правого класса другой.

r из левого класса $\{ I_n \} \exists n: r < a_n$
\begin{enumerate}
    \item[i.] r из правого класса $\{ I_n` \} \Rightarrow \exists n`: \forall n > n`\;\; a_n` \leq b_n` < r < a_n \leq a_m$\\
    \item[ii.] r из центрального класса $\{ I_n` \}$\\
    $\exists n: r < a_n$ Пусть $\eps = a_n - r(>0)$\\
    $\exists n_\eps: \forall m > n_\eps |b_m`-a_m`| < \cfrac{\eps}{2} \Rightarrow [a_m`,b_m`]$ на расстоянии не меньше $\cfrac{\eps}{2}$ от $a_n$
\end{enumerate}

\textbf{2 вариант доказательства в обратную сторону.}\\
$\Leftarrow: a_n - a_n`\; \xrightarrow{n\rightarrow \infty} 0 \Rightarrow \{ I_n \} \sim \{ I_n` \}$\\
а) совпадение левых классов $r \in $ левый класс для $\{I_n\}$
\includegraphics[width=0.4\linewidth]{images/Дедекиндовы сечения/Доказательство Т1.2.png}\\
$\exists n: r < a_n$\\
$\exists n_\eps: \forall n>n_\eps \;\;|a_n-a_n`| < \cfrac{\eps}{2} \Rightarrow r < a_n` \Rightarrow r\ \in $ левый класс для $\{I_n`\}$ \\
б) правые классы аналогично
\end{tcolorbox}

\subsubsection{Ограничение на количество элементов центрального класса.}
Если r из центрального класса, то $\forall n: \; a_n \leq r \leq b_n$\\
А если $\exists r`,r \in $ центральный класс $r<r`$, то $a_n \leq r < r` \leq b_n$. Тогда длина отрезка не может быть меньше длины отрезка $[r,r`]$ значит она не стремится к 0.\\
В центральном классе может быть либо 1 число либо 0.
\subsubsection*{Пример 2-х последовательностей стягивающихся к 0}
$\{I_n\} = [\cfrac{1}{2n},\cfrac{1}{2n}]\;\; \{I_n`\} = [\cfrac{1}{2n+1},\cfrac{1}{2n+1}] \;\; \{I_n``\} = [0,\cfrac{1}{2n+1}]$ Они определяют 1 и то же число.
\subsubsection*{Пример 2-х последовательностей стягивающихся к $\sqrt{2}$}
$[\sqrt{2} - \cfrac{1}{n}, \sqrt{2} + \cfrac{1}{n}]$
\subsubsection{Действительные числа}
\begin{tcolorbox}
\textit{\textbf{Действительные числа} - это вложенные стягивающиеся отрезки с рациональными концами. Числа равны, если последовательность $\{I_n\} \sim \{I_n`\}$  }\\
\textit{\textbf{Действительные число} - отождествляется с дедекиндовым сечением, порождённым $\{[a_n, b_n]\}$ . Числа равны, если последовательность $\{[a_n, b_n]\} \sim \{[a_n`, b_n`]\}$  }
\end{tcolorbox}

\subsubsection*{Ограничение на количество элементов центрального класса.}
Если r из центрального класса, то $\forall n: \; a_n \leq r \leq b_n$\\
А если $\exists r`,r \in $ центральный класс $r<r`$, то $a_n \leq r < r` \leq b_n$. Тогда длина отрезка не может быть меньше длины отрезка $[r,r`]$ значит она не стремится к 0.\\
В центральном классе может быть либо ё число либо 0.
\subsubsection*{Пример 2-х последовательностей стягивающихся к 0}
$\{I_n\} = [\cfrac{1}{2n},\cfrac{1}{2n}]\;\; \{I_n`\} = [\cfrac{1}{2n+1},\cfrac{1}{2n+1}] \;\; \{I_n``\} = [0,\cfrac{1}{2n+1}]$ Они определяют 1 и то же число.
\subsubsection*{Пример 2-х последовательностей стягивающихся к $\sqrt{2}$}
$[\sqrt{2} - \cfrac{1}{n}, \sqrt{2} + \cfrac{1}{n}]$
\subsection{Лемма об отделимости.}

\subsection{Точная верхняя и нижняя грани ограниченных множеств из В. Теорема Вейерштрасса о существовании точной верхней грани ограниченного сверху множества как следствие леммы об отделимости (принцип полноты В по Вейерштрассу).}

\begin{tcolorbox}
    \textbf{Принцип полноты множества по Вейерштрассу}
    Принцип полноты множества по Вейерштрассу означает, что любое ограниченное сверху множество имеет точную верхнюю грань     
\end{tcolorbox}

\begin{tcolorbox}
    \textbf{Определение}
    $\{a_n\}$ монотонна, если возрастает/ строго возрастает/убывает/строго убывает.
\end{tcolorbox}

\begin{tcolorbox}
    \textbf{Теорема по Вейерштрассу}
    \begin{enumerate}
        \item $\{a_n\} \uparrow \; \Rightarrow a_n \to sup\{ a_n \}$
        \item $\{a_n\} \downarrow \; \Rightarrow a_n \to inf\{ a_n \}$
    \end{enumerate}
\end{tcolorbox}

\begin{tcolorbox}
    \textbf{Определение}
    $sup \{a_n\}$ - это $sup$ множества членов последовательности.
\end{tcolorbox}

\begin{tcolorbox}[title=Доказательство полноты $\R$ по Вейерштрассу]
    \textbf{Доказательство}
    \begin{enumerate}
        \item Ограничено сверху $\exists M: \forall n a_n \leq M \Rightarrow \exists sup\{a_n\} = M \in \R$\\
        По определению предела $\forall \varepsilon > 0 \exists n_\varepsilon: \forall n > n_\varepsilon |a_n - M| < \varepsilon $\\
        $\Leftrightarrow M-\varepsilon<a_n<M+\varepsilon$ т.к. $M=sup\{a_n\}$, то $a_n \leq M$ надо проверить $M-\varepsilon<a_n\leq M$\\
        M - наименьшая верхняя грань $\Rightarrow \exists n_\varepsilon: \forall n > n_\varepsilon \;\;\;\;\; M-\varepsilon \leq a_{n_\varepsilon} \leq a_n \leq M $\\
        т.е. по определению 
        \[M = \lim_{n\to\infty} a_n\]
        \textbf{Следствие} $\{a_n\}$ - монотонна $\{a_n\}$ - сходится $\Leftrightarrow \{a_n\}$ - ограничена.
    \end{enumerate}
\end{tcolorbox}

\subsection{Последовательности стягивающихся отрезков с действительными концами. Теорема Кантора о стягивающихся отрезках с действительными концами (принцип полноты ® по Кантору).}

\subsection{Полнота К по Дедекинду как следствие принципа стягивающихся отрезков.}

\subsection{Счётность множества, рациональных чисел и несчётность множества действительных чисел.}

