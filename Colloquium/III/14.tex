\subsection{Числовые ряды. Абсолютная и условная сходимость числовых рялов. Критерий Коши сходимости ряда. Необходимое условие сходимости ряда. Признак сравнения.}

\subsubsection{Определения и элементарные факты.}
$ \{a_k\} $ - ЧП
\[ \{a_k\} \to S_n = \sum^n_{k=1}a_k \]
\textbf{Определения}
\begin{tcolorbox}
    \textbf{Определяем бесконечную сумму}.\\
    $\sum^\infty_{k=1} a_k$ - \textbf{ряд}\\
    $a_k$ - \textbf{элемент ряда (общий член)}.\\
    $S_n$ - \textbf{n-ая частичная сумма ряда}.\\
    Если $S_n$ - сходится, то ряд ($\sum^\infty_{1} a_k$) называется \textbf{сходящимся}, $S:=\lim_{n\to\infty} s_n$ - \textbf{суммой ряда}: $\sum_1^\infty = S$.\\
    Если $\{S_n\}$ расходится, то $\sum^\infty_{1} a_k$ - \textbf{расходится}.
\end{tcolorbox}
\subsubsection{Теорема 1.}
\[ \sum_1^\infty a_k \pm \sum_1^\infty b_k = \sum_1^\infty (a_k \pm b_k) \]
\subsubsection{Теорема 2. Критерий Коши о сходимости ряда.}
$ \sum_1^\infty a_k$ - сходится $\Leftrightarrow \forall \eps > 0\;\; \exists n_\eps : \forall n > n\eps \forall p \;\;\; |S_{n+p} - S_n| < \eps $
\[ |\sum_1^{n+p} a_k - \sum_1^n a_k| = |\sum_{n+1}^{n+p} a_k| < \eps \]
\subsubsection*{Следствие 1. Изменение конечного числа членов ряда не влияет на сходимость.}
Сумма конечно может измениться, но на сходимость это не влияет.
\subsubsection*{Следствие 2. Необходимое условие сходимости ряда.}
Можно считать определением: Если $\sum_1^\infty a_k$ сходится $\Rightarrow a_k \xrightarrow[k\to\infty]{} 0$\\
\begin{tcolorbox}[title=Доказательство]
    $\sum_1^\infty a_k$ - сходится $\forall \eps > 0\; \exists n_\eps:\; \forall n > n_\eps\;\; \forall p \ |\sum_{n+1}^{n+p} a_k| < \eps \Leftrightarrow |a_{n+1}| < \eps$\\
    При p = 1, то есть $\{a_n\}$ - б.м. по определению.
\end{tcolorbox}
\subsubsection*{Пример 1.}
Если |q| < 1
\[ \sum_1^\infty q^k, S_n = 1+...+q^n = \cfrac{1-q^{n+1}}{1-q} \xrightarrow[n\to\infty]{} \cfrac{q}{1-q} \]
\subsubsection*{Пример 2.}
$\sum_1^\infty \cfrac{1}{n}$\\
a, b, c положительные. c - среднее гармоническое a и b, если $\cfrac{2}{c} = \cfrac{1}{a} + \cfrac{1}{b}$\\
$a = \cfrac{1}{n-1}, b = \cfrac{1}{n+1}$\\
$\cfrac{2}{c} = (n - 1) + (n + 1) = 2n$\\
Каждый элемент является средним гармоническим 2-х соседий. По критерию Коши ряд - расходящийся.
\[ |\sum_{n+1}^{n+p} \cfrac{1}{k}| = \cfrac{2}{n+1} + ... + \cfrac{1}{n+p} \geq \cfrac{p}{n+p}\]
Пусть p = n. $\cfrac{p}{n+p} = \cfrac{1}{2}$.
\subsubsection*{\textbf{Пример 3.}}
\[ 1 - 1 + \cfrac{1}{2} - \cfrac{1}{2} + \cfrac{1}{3} - \cfrac{1}{3} + ...\]
$S_{2n} = 0$\\
$S_{2n+1} = \cfrac{1}{n+1} \to 0$\\
А теперь давайте мухлевать. Переставим сумму ряда. Шоу ИМПРОВИЗАЦИЯ!!!\\
$(1+\cfrac{1}{2}) - 1 + (\cfrac{1}{3} + \cfrac{1}{4} +\cfrac{1}{5} + \cfrac{1}{6} + \cfrac{1}{7} + \cfrac{1}{8} + ... + \cfrac{1}{11}) - \cfrac{1}{2}$\\
Берём много положительных слагаемых и вычитаем меньшее по модулю число. Из-за этого ряд расходится.
\subsubsection*{\textbf{Пример 4.}}
\[ \sum^\infty_{0} (-1)^k = (1-1) + (1-1) + ... \]
\[ \sum^\infty_{0} (-1)^k = 1 - (1-1) - (1-1) - ... \]

\subsubsection{Абсолютно сходящийся ряд.}
\begin{tcolorbox}
    \textbf{Определение.} Если $\sum^\infty_{1} |a_k|$ сходится, то $\sum^\infty_{1} a_k$ сходится абсолютно.
\end{tcolorbox}
\subsubsection{Теорема 1. Если ряд сходится абсолютно, то ряд сходится.}
\begin{tcolorbox}[title=Доказательство (Критерий Коши)]
    По критерию Коши, т.к. $\sum^\infty_{1} |a_k|$ сходится, то $\forall \eps > 0\; \exists n_\eps:\; \forall n > n_\eps\; \forall p\;\; |\sum^{n+p}_{n+1} a_k| \leq |\sum^{n+p}_{n+1} |a_k|| < \eps \Rightarrow$ по критерию Коши $\sum a_k$ сходится.
\end{tcolorbox}
\subsubsection*{\textbf{Пример 1.}}
$1 - 1 + \cfrac{1}{2} - \cfrac{1}{2} + ... $ - сходится, но не абсолютно.
\subsubsection*{Определение.}
Если $\sum^{\infty}_{1} a_k$ сходится, а $\sum^{\infty}_{1} |a_k|$ расходится, то $\sum a_k$ сходится условно.
\subsubsection{Теорема 2.}
$\sum^{\infty}_{1} a_k, \forall k\; a_k \geq 0$\\
$\sum^{\infty}_{1} a_k$ сходится $\Leftrightarrow \{S_N\}$ ограничена.
\begin{tcolorbox}[title=Доказательство]
    \[\forall n\; S_{n+1} = \sum^{n+1}_{1} a_k \geq \sum^{n}_{1} a_k = S_n \]
    $S_n \uparrow$ возрастающая $\{S_n\}$ сходится $\Leftrightarrow$ ограничена.
\end{tcolorbox}
\subsubsection{Теорема 3. Признак сравнения.}
Для комплов не годится.\\
$\sum^{\infty}_{1} a_k,\;\; \sum^{\infty}_{1} b_k,\;\; \forall k\;\; a_k \geq \underline{b_k \geq 0}$\\
Тогда:
\begin{enumerate}
    \item Если $\sum^{\infty}_{1} a_k$ сходится $\Rightarrow \sum^{\infty}_{1} b_k$ сходится.
    \item Если $\sum^{\infty}_{1} a_k$ расходится $\Rightarrow \sum^{\infty}_{1} b_k$ расходится.
\end{enumerate}
\begin{tcolorbox}[title=Доказательство]
Следствие критерия сходимости ряда с неотрицательными членами.\\
\begin{enumerate}
    \item $A_n = \sum^{n}_{1} a_k,\;\; B_n = \sum^{n}_{1} b_k $\\
    $A_n \uparrow,\; B_n \uparrow$ и $A_n \geq B_n$ ($A_n$ можарирует $B_n$)\\
    Если $\sum a_k$ сходится $\Rightarrow \{A_n\}$ ограничена сверху $\Rightarrow \{B_n\}$ ограничена сверху $\Rightarrow \sum b_k$ сходится.
\end{enumerate}
\end{tcolorbox}

\subsubsection*{Следствие. (Признак сравнения).}
$\sum a_k,\; \sum b_k \;\; \forall k\; a_k \geq |b_k| > 0$. Не отрицательность $a_k$.\\
Сходимость $\sum a_k \Rightarrow$ сходимость $\sum b_k$ (абсолютная).
\subsubsection*{Пример 1.}
\[\sum^{\infty}_{1} \cfrac{\sin n}{n^2}\]
\[|\cfrac{\sin n}{n^2}| \leq \cfrac{1}{n^2} < \cfrac{1}{(n-1)n}, n \not= 1\]
\[ \cfrac{1}{(n-1)n} = \cfrac{1}{n-1} - \cfrac{1}{n} \]
\[ \sum^{\infty}_{2} \cfrac{1}{n-1)n} = \sum^{\infty}_{2}( \cfrac{1}{-1} - \cfrac{1}{n}) \]
\[ S_n = \sum^{n}_{2} (\cfrac{1}{k-1} - \cfrac{1}{k}) = (\cfrac{1}{2-1} - \cfrac{1}{2}) + (\cfrac{1}{2} - \cfrac{1}{3}) + ... \]

