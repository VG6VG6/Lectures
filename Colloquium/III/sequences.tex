\subsection{Свойства сходящихся последовательностей (сходимость постоянной последовательности, единственность предела, ограниченность сходящейся последовательности).}

\subsubsection{Предел последовательности}
\begin{tcolorbox}
\textbf{Определение.} Число $a$ называется \textit{пределом последовательности} $\{a_n\}$, если
$$\forall \eps > 0\ \exists N_\eps \in \N\ \forall n > N_\eps: |a_n - a| < \eps$$
Обозначение: $\lim_{n\to\infty} a_n = a$ или $a_n \to a$ при $n \to \infty$.
\end{tcolorbox}

\subsubsection{Сходимость постоянной последовательности}
\begin{tcolorbox}
    \textbf{Теорема 1.} Если $\exists N: \forall n > N \;\; a_n = a \Rightarrow a_n\to a$
\end{tcolorbox}
\begin{tcolorbox}[title=Доказательство]
    \[ \forall \varepsilon>0 \;\;\forall n>N \;\; |a_n - a| = 0 < \varepsilon \Rightarrow a_n\to a\] по определению 
\end{tcolorbox}

\begin{tcolorbox}
    \textbf{Теорема 2.} Если $ a_n\to a,\;\; b_n\to b \;\;\;\forall n > N \;\; a_n \leq b_n \Rightarrow a\leq b $
\end{tcolorbox}
\begin{tcolorbox}[title=Доказательство]
    \textbf{От противного:}\\ 
    Пусть $ a < b$
\end{tcolorbox}
\textbf{НАДО ПОСМОТРЕТЬ ПРЕДЫДУЩИЕ ЛЕКЦИИ. ЭТО ОТТУДА.}

\subsubsection{Теорема о единственности предела.}
\textbf{Теорема:} Последовательность не может иметь более одного предела.

\begin{tcolorbox}[title=Доказательство теоремы, breakable]
\textbf{Доказательство от противного:}
Предположим, что последовательность $\{a_n\}$ имеет два различных предела: $a_n \to A$ и $a_n \to B$, где $A \neq B$.
Пусть $\eps = \frac{|A - B|}{4} > 0$. Тогда по определению предела:
\begin{itemize}
    \item $\exists N_1: \forall n > N_1: |a_n - A| < \eps$
    \item $\exists N_2: \forall n > N_2: |a_n - B| < \eps$
\end{itemize}
Возьмём $n > \max(N_1, N_2)$. Тогда выполняются оба неравенства. Оценим разность $|A - B|$:
$$|A - B| = |(A - a_n) + (a_n - B)| \leq |A - a_n| + |a_n - B| < \eps + \eps = 2\eps = \frac{|A - B|}{2}$$
Получили противоречие: $|A - B| < \frac{|A - B|}{2}$. Следовательно, наше предположение неверно, и предел единственен.
\end{tcolorbox}

\subsubsection{Ограниченные и сходящиеся последовательности}

\begin{tcolorbox}
\textbf{Определение}
$\{a_n\}$  \textit{ограничена}, если $\exists M>0: \forall n\ |a_n|<M,\ a_n, M, n \in Q$
\end{tcolorbox}

\begin{tcolorbox}
\textbf{Определение}
$\{a_n\}$  \textit{не ограничена}, если $\forall M>0: \exists n\ |a_n|\geq M,\ a_n, M, n \in Q$ 
\end{tcolorbox}

\begin{tcolorbox}
\textbf{Теорема 2} \textit{Любая сходящаяся последовательность ограничена} \\
Если $\{a_n\}$ сходиться $\Rightarrow \{a_n\}$ ограничена
\end{tcolorbox}

\begin{tcolorbox}[title=Доказательство Т2, breakable]
\[\eps := 1\ \exists N: \forall n > N\ \ |a_n-a| < 1 \Leftrightarrow -1<a_n-a<1 \Leftrightarrow a-1 < a_n < a+1  \]
\[ M:= max\{ |a_1|, |a_2|, ..., |a_N|, |a-1|, |a+1| \} + 1\]
$\Rightarrow \{a_n\}$ - ограничена (сверху)
\end{tcolorbox}

\begin{tcolorbox}
\textbf{Определение}
$\{a_n\}$  \textit{ограничена сверху, если} $\exists M: \forall n$ $a_n < M$
\end{tcolorbox}

\begin{tcolorbox}
\textbf{Определение}
$\{a_n\}$  \textit{ограничена снизу, если} $\exists m: \forall n$ $a_n < m$
\end{tcolorbox}

Пример:
\begin{enumerate}
    \item $\{\cos n\}\ |\cos n| \leq 1$
    \item $\{n\}$ ограничена снизу но не сверху (0, 1, ...)
\end{enumerate}

