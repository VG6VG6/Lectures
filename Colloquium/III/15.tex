\subsection{Признаки абсолютной сходимости рядов Даламбера и Коши.}

\subsubsection{Теорема 4. Признак Коши.}
\[ \sum_1  a_k\]
\[q = \overline{lim} \sqrt[k]{|a_k|} \]
Тогда:
\begin{enumerate}
    \item $q < 1 \Rightarrow$ абсолютно сходится.
    \item q > 1 расходится
    \item q = 1 ?
\end{enumerate}
\begin{tcolorbox}[title=Доказательство]
    Сравнение с геометрической прогрессией.\\
    \begin{enumerate}
        \item $ 0 \leq q < p < 1 $
        \includegraphics[width=0.5\linewidth]{Ряды/Признак Коши Док-во.png}\\
        $\exists n_p: \forall n>n_p$\\
        $\sqrt[k]{a_k} < p \uparrow \Leftrightarrow |a_k| < p_k$\\
        $\sum^{\infty}_{n+1} p^k$ геометрическая прогрессия с положительным знаком |x|.\\
        $\Rightarrow \sum^{\infty}_{1} a_k$ сходится абсолютно (по признаку сравнения)
        \item q = $\overline{lim} \sqrt[k]{a_k} > 1$\\
        $1 > p > 1$\\
        $\forall$ окрестности $q$ бесконечно много членов $\sqrt[k]{|a_k|} $
    \end{enumerate}
\end{tcolorbox}
\textbf{Замечание.} Признак Коши бесполезно использовать, если ряд не похож на геометрическую прогрессию.\\
\textbf{Запомните.} Признак коши достаточное условие абсолютной сходимости.\\
\textbf{Следствие} Если $ \sqrt[n]{|a_n|} \leq q < 1 \Rightarrow \exists q,N\; \forall n > N: \;  \sum a_n$ сходится абсолютно

\textbf{Пример.} $ \sum^\infty_{n=1} (2+(-1)^n)^n \cdot z^n$\\
\[
	\sqrt[n]{((2+(-1)^n)^n z^n} = (2+(-1)^n)|z|\\
	n = 2k: \sqrt[n]{|a_n|} = 3|z| = \overline{\lim} \sqrt[n]{|a_n|} \\
	n = 2k+1: \sqrt[n]{|a_n|} = |z|\\ 
\]
Если: 
	$3|z| < 1$ сходится $|z| < \cfrac{1}{3}$\\
	$3|z| > 1$ расходится $|z| > \cfrac{1}{3}$\\
	$3|z| = 1 |z| = \cfrac{1}{3} $

\textbf{Теорема 5. Признак д-Аламбера}\\
!Он всегда слабее признака Коши.
\begin{tcolorbox}
	\[
		\sum_{n=1}^{\infty}: |\cfrac{a_{n+1}}{a_{n}}| \to q
	\]
\begin{enumerate}
	\item $q < 1 \Rightarrow$ абсолютно сходится
	\item $q < 1 \Rightarrow$ расходится
	\item $q = 1 \Rightarrow$ Ничего не даёт
\end{enumerate}
\end{tcolorbox}

\begin{tcolorbox}[title=Доказательство.]
	\begin{enumerate}
		\item  Сравнение с геометрической прогрессией.\\
		$\exists n_p: \forall n > n_p \;\; |\cfrac{a_{n + 1}}{a_n}| < p$\\
		Пусть для $\forall n$\\
		$ a_{n+1} = \cfrac{a_{n+1}}{a_n} \cdot  \cfrac{a_{n}}{a_{n-1}} \cdot ... \cdot \cfrac{a_{2}}{a_1} $
		$ |a_{n+1}| = |\cfrac{a_{n+1}}{a_n}| \cdot  |\cfrac{a_{n}}{a_{n-1}}| \cdot ... \cdot |\cfrac{a_{2}}{a_1}| \cdot |a_1| < p^n |a_1| $\\
		$ \displaystyle\sum_{}^{} p^n |a_1|$ сходится (геометрическая прогрессия)
	\item $\exists N: |\cfrac{a{_n+1}}{a_n}| > 1\;\; \forall n > N$\\
		Пусть $ \forall N\; |\cfrac{a_{n+1}}{a_n}| > 1 $\\
		$ |a_{n+1}| > |a_{n}| > |a_{n-1}| > ... |a_{1}| > 0 \Rightarrow a_n \not\to 0$
	\item $ \displaystyle\sum_{1}^{\infty} \cfrac{1}{n}\;\; |\cfrac{a_{n+1}}{n}| = \cfrac{n}{n+1} \to 1$\\
		$ \displaystyle\sum_{1}^{\infty} \cfrac{1}{n^2}\;\; |\cfrac{a_{n+1}}{n}| = \cfrac{n^2}{(n+1)^2} \to 1$\\
 	\end{enumerate}
\end{tcolorbox}

\textbf{Следствие.} $\forall n > N:\; |\cfrac{a_{n+1}}{a_n}| \leq q < 1 \Rightarrow \sum a_n$ сходится абсолютно.

\begin{tcolorbox}[title=Доказательство следствия]
	Пусть $ \forall n\; |\cfrac{a_{n+1}}{a_n}| \leq q < 1 $\\
$ |a_{n+1}| = |\cfrac{a_{n+1}}{a_n} \cdot |\cfrac{a_{n}}{a_{n-1}} \cdot ... \cdot |\cfrac{a_{2}}{a_1} \cdot |a_1| \leq q^n |a_1|$ Т.к. $|q| < 1$, то $\displaystyle\sum_{}^{} q^n|a_1|$ сходящаяся геометрическая прогрессия.
\end{tcolorbox}
\textbf{Вопрос на 5:} Если можно исследовать по Деламберу то можно и по Коши.\\

\textbf{Пример}
	$\displaystyle\sum_{1}^{\infty} \cfrac{1}{(3+)-1)^n)^n}$\\
	$\sqrt[n]{|a_n|} = \cfrac{1}{3+(-1)^n} = \cfrac{1}{4}, n = 2k$\\
	$\sqrt[n]{|a_n|} = \cfrac{1}{3+(-1)^n} = \cfrac{1}{2}, n = 2k+1$\\
	$\sqrt[n]{|a_n|} \leq \cfrac{1}{2} < 1$

	$|\cfrac{a_{n+1}}{a_{n}} = \cfrac{\cfrac{1}{2^{2k+1}}{\cfrac{1}{4^{2k}}}} = \dots$\\
\textbf{Пример} 
    $\sum_{1}^{\infty} \cfrac{1}{(3+)-1)^n)^n}$\\
	$\sqrt[n]{|a_n|} = \cfrac{1}{3+(-1)^n} = \cfrac{1}{4}, n = 2k$\\
	$\sqrt[n]{|a_n|} = \cfrac{1}{3+(-1)^n} = \cfrac{1}{2}, n = 2k+1$\\
	$sqrt[n]{|a_n|} \leq \cfrac{1}{2} < 1$

$
	|\cfrac{a_{n+1}}{a_{n}} = \cfrac{\cfrac{1}{2^{2k+1}}}{\cfrac{1}{4^{2k}}} = \cfrac{4^{2k}}{2^{2n+1}} = \cfrac{1}{2}2^{2k}
    n = 2k
$

\[
	|\cfrac{a_{n+1}}{a_{n}}| = \cfrac{\cfrac{1}{2^{2k+1}}}{\cfrac{1}{4^{2k}}} = \cfrac{4^{2k}}{2^{2n+1}} = \cfrac{1}{2^{n+1}}\\
    n = 2k+1
\]

\textbf{Замечание.} $|\cfrac{a_{n+1}}{a_n}| \leq 1 < 1$, а не $|\cfrac{a_{n+1}}{a_n}| < 1$

\textbf{Пример.} \[ \sum_0^\infty \cfrac{z^n}{n!} \]
\[ |\cfrac{a_{n+1}}{a_n}| = \cfrac{\cfrac{z^{n+1}}{(n+1)!}}{\cfrac{z^n}{n!}} = \cfrac{|z|}{n+1} \to 0 \]

