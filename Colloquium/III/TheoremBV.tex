\subsection{Теорема Больцано-Вейерштрасса.}

$n_k$ - строго возрастает последовательность натуральных чисел $\Rightarrow n_k\geq k$ .

\begin{tcolorbox}
    \textbf{Определение} $\forall \{a_n\} \; \{a_n\}$ - подпоследовательность $\{a_n\}$
\end{tcolorbox}
$\{a_n\} = 2^n$, тогда 2, 16, 64, 128 - подпоследовательность\\
$\begin{cases}
    2, \textbf{3}, 4, 16, ...\\
    16, \textbf{2}, 32, ...     
\end{cases}$ - не подпоследовательности\\

\textbf{Теорема Больцано-Вейерштрасса}
\begin{tcolorbox}[]
    $ \forall $ ограниченой $\{a_n\} \exists $сходящаяся подпоследовательность $ \{a_n\} $
\end{tcolorbox}

\begin{figure}[h]
    \centering
    \includegraphics[width=0.5\linewidth]{Т Б-В.png}
\end{figure}

\begin{tcolorbox}[title=Доказательство]
    Прицип вложенных отрезков.\\
    $\exists [c,d]: \forall n\; a_n\in[c,d]$
    \begin{enumerate}
        \item $[c_1,d_2] = [c,d] \;\; \forall a_{n_1} \in [c_2,d_1]$
        \item $b_1 = \cfrac{c_1+d_1}{2}$
                $[c_2,d_2]$ тот из $[c_1,b_1]$ и $[b_1,d_1]$ на котором содержится бесконечно много членов последовательности $\{a_n\}$\\
        $a_{n_2}: a_{n_2} \in [c_2,d_2], n_2>n_1$
        \item $b_2 = \cfrac{c_2+d_2}{2}$\\
        $[c_3,d_3]$ тот из $[c_2,b_2]$ и $[b_2,d_2] ...$\\
        $a_{n_3}: a_{n_3} \in [c_3,d_3], n_3>n_3$

        $\{a_n\} \;\;a_{n_k} \in [c_k,d_k], n_k > n_{k-1}$
        $[c_1,d_1] \supset [c_2,d_2] \supset ... \supset [c_k,d_k] \supset ...$\\
        длина $[c_k,d_k] = \cfrac{1}{2^{k-1}}$
        длина $[c_1,d_1] \to 0$
        $\Leftrightarrow \exists a` = U[c_k,d_k]$\\
        $\forall k\;\; |a_n-a`| \leq \cfrac{1}{2^{k-1}}$ - длина $[c_1,d_1] \to 0$
        $a_{n_k}, a`\in [c_k,d_k] \Rightarrow a_{n_k} \to 0`$
    \end{enumerate}
\end{tcolorbox}

\begin{tcolorbox}
    \textbf{Определение} Частичный предел $\{a_n\}$ - предел $\forall$ сходящейся подпоследовательности $\{a_n\}$
\end{tcolorbox}
\begin{tcolorbox}[title=Следствие из теоремы]
    $\forall \{a_n\} \exists$ подпоследовательность $\{a_{n_k}\}$ которая имеет либо конечное либо бесконечное число пределов.
\end{tcolorbox}
\begin{tcolorbox}[title=Доказательство]
    Если в $\{a_n\} \exists \{a_{n_k}\} a_{n_k} \to a` \in \R$\\
    Если такого нет, то по Теореме Б-В $\{a_{n}\}$ - не ограничена сверху или снизу.\\

    Если $\{a_{n}\} $ - не ограничена сверху:
    \begin{enumerate}
        \item 1 - не верхняя грань $\{a_{n}\}: a_{n_1}: a_{n_1} > 1$
        \item 2 - не верхняя грань $\Rightarrow \exists n_2: a_{n_2}>2, n_2>n_1$
        \item[...]
        \item [k.] $\exists n_k: a_{n_k}>max\{k, a_1, a_2, ..., a_{n_{k-1}}\}, n_k>n_{k-1}$
    \end{enumerate}
\end{tcolorbox}
