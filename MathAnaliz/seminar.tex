\documentclass[12pt, paper]{article}
\usepackage{graphicx}
\usepackage{amsmath, amssymb}
\usepackage[english, russian]{babel}
\usepackage[T2A]{fontenc}
\usepackage[utf8]{inputenc}
\usepackage[margin=2cm]{geometry}
\usepackage{tcolorbox}
\usepackage{caption}
\usepackage{enumitem}

\tcbuselibrary{breakable}
\tcbset{
  width=0.9\textwidth,
  halign=justify,
  center,
  breakable,
  colback=white
}

\title{Семинары по мат аналу}
\date{Октябрь 2025}
\newcommand{\R}{\mathbb{R}}
\newcommand{\N}{\mathbb{N}}
\newcommand{\eps}{\varepsilon}

\begin{document}

\maketitle
\newpage
\section{Пределы}
\subsection{Доказано, что (на зачёте уметь доказать):}
$ \cfrac{a^k}{a^n} \to 0$, при $a > 1$
\section{Ряды}
\subsection{Определение сходимости ряды}
\[ \sum_{k=1}^\infty a_k\]
\[ S_n = \sum_{k=1}^\infty a_k\]

\begin{tcolorbox}
\textbf{Определение.} \\
Если $S_n$ - сходится, при $n \to \infty \Rightarrow  \sum_{k=1}^\infty $ - сходится 
\end{tcolorbox}

\subsection{Критерий Коши}
\begin{tcolorbox}
    \[ \forall \varepsilon > 0\;\; \exists n_\varepsilon\;\; \forall n,m > n_\varepsilon \;\; |S_n-S_m| < \varepsilon\]
    \[ \forall \varepsilon > 0\;\; \exists n_\varepsilon\;\; \forall n > n_\varepsilon,\forall p \in \N |\sum_{k=n+1}^{n+p} a_k|< \varepsilon \]    
\end{tcolorbox}


\[a_n = S_n - S_{n-1}, n\to \infty\]
\[ S_n \to S, S_{n-1} \to S \Rightarrow a_n \to 0\]

\textbf{Пример} 
\[\sum_{k=1}^\infty  \cfrac{1}{n}\]
\[ \forall \varepsilon > 0\;\; \exists n_\varepsilon\;\; \forall n > n_\varepsilon \forall p \in \N\]
\[ |\sum_{n+1}^{n+p} a-n| < \varepsilon\;\;\; \cfrac{1}{1+n} + \cfrac{1}{2+n} + \dots + \cfrac{1
}{n+p} \geq \cfrac{1}{n+p} + \dots \cfrac{1}{n+p} = \cfrac{p}{n+p} =^{p=n}= \cfrac{1}{2} \]
Противоречие:
\[\sum_{k=n+1}^{2n} \geq \cfrac{1}{2}\]
\textbf{Пример} 
\[\sum_{k=1}^\infty  \cfrac{1}{\sqrt{n}}\]
\begin{tcolorbox}
    \[ \exists \varepsilon > 0\;\;\forall n, p=3n \]
    \[ |\sum_{k=n+1}^{n+p}a_k| \geq 1 \]
\end{tcolorbox}
\[ \sum_{k=n+1}^{n+3n} \cfrac{1}{k} = \sum_{k=n+1}^{2n} + \sum_{k=2n+1}^{4n} \]

\begin{tcolorbox}[title=Ряд геометрической прогресии]
    \[ \sum_1^n q^k = \cfrac{1-q^{n+1}}{1-q} - \xrightarrow{n\to \infty} \cfrac{1}{1-q} \]
\end{tcolorbox}

$ \sum_1^\infty |a_k| $ - сходится $\Rightarrow \sum_1^\infty a_k $ - тоже сходится \\
$ \forall \varepsilon > 0\;\; \exists n_\varepsilon: \forall n > n_\varepsilon, \forall p |\sum^{n+p}_{n+1} a_k| \leq \sum^{n+p}_{n+1} |a_k| = |\sum^{n+p}_{n+1} |a_k|| < \varepsilon $

\textbf{Пример. } $ \sum_1^\infty\cfrac{(-1)^{k+1}}{k} $
\[ S_{2n} = a_1 + a_2 + \dots + a_{2n} = 1 - \cfrac{1}{2} + \cfrac{1}{3} - \cfrac{1}{4} + \dots + \cfrac{1}{2n-1} - \cfrac{1}{2n} = (1 - \cfrac{1}{2}) + (\cfrac{1}{3} - \cfrac{1}{4}) + \dots + (\cfrac{1}{2n-1} - \cfrac{1}{2n}) = \]
\[ \cfrac{1}{2} + \cfrac{1}{3\cdot4} + \cfrac{1}{5\cdot6} + \dots | \cfrac{1}{(2n-1)2n} < \cfrac{1}{1^2}+ \cfrac{1}{2^2}+\cfrac{1}{3^2}+\dots+\cfrac{1}{n^2} = \sum_{k=1}^{n} \cfrac{1}{k^2} < \cfrac{\pi^2}{6} \]
Тогда 
$ \sum_{k=1}^{2n} \cfrac{1}{k^2} $ - имеет предел, а значит и $ \sum_1^{2n}\cfrac{(-1)^{k+1}}{k} $ имеет предел

Докажем для нечётных
\[ S_n \to S \]
\[ S_{n+1} = S_n + \cfrac{1}{2n+1} \xrightarrow{n\to\infty} S_n \]
Тогда $ \sum_1^\infty\cfrac{(-1)^{k+1}}{k} $ имеет предел


\begin{tcolorbox}
    $ \sum a_k -$ сходится $ \sum b_k: \forall k |b_k| \leq a_k \Rightarrow \sum b_k$ - сходится
\end{tcolorbox}

\textbf{Доказать, что ряд сходится} $ \sum \cfrac{n^3\cdot cos\; n + 3}{n^5 + 10n^3 \cdot sin\;n} $\\
Считаем доказаным, что \\
$ \sum \cfrac{1}{n} $ - расходится\\
$ \sum \cfrac{1}{n^2} $ - сходится\\
$ \sum q^n,\;\; |q| < 1 $ - сходится\\

\[ \sum \cfrac{n^3\cdot cos\; n + 3}{n^5 + 10n^3 \cdot sin\;n} = \cfrac{cos\;n+\cfrac{3}{n^3}}{n^2+10\cdot sin\;n} \]
\[ |\cfrac{cos\;n+\cfrac{3}{n^3}}{n^2+10\cdot sin\;n}| \leq \cfrac{4}{n^2-10} \]
При $ n > 3$
\[ \cfrac{4}{n^2-10} < \cfrac{4}{\frac{n^2}{2}} \]

\subsection{Признак Аламбера}
\begin{tcolorbox}[title=признак Аламбера]
    \[ \sum_{k=1}^{\infty} a_k \;\;\;\forall k\;\; |\cfrac{a_{k+1}}{a_k}| \leq q < 1 \]
    Значит ряд сходится.
\end{tcolorbox}

\textbf{Пример.} $\sum^\infty_0 \cfrac{a^n}{n!}$
По признаку Аламбера: 
\[ |\cfrac{\frac{a^{n+1}}{(k+1)!}}{\frac{a^n}{k!}}| = |\cfrac{a}{k+1}| \xrightarrow{k\to \infty} 0 < 1\Rightarrow \sum^\infty_0 \cfrac{a^n}{n!} \]
Ряд сходится

\subsection{Радикальный признак Коши}
\begin{tcolorbox}[title=Признак Коши]
    \[ \sum aьь_k \]
    \[ \sqrt[k]{|a_k|} \leq q < 1 \]
\end{tcolorbox}

\subsection{Необходимое условие сходимости ряда}
$ \sum a_n$ сходится $\Leftrightarrow  a_n \to 0 $

$a_1 \geq a_2 \geq ... \geq a_n \geq ... \geq 0$\\
$\sum a_k$ сходится $\Leftrightarrow \sum 2^k a_{2^k}$

\section{О малая и О большое}
\[ g(x) = \overline{\overline{o}}(f(x)),\ x \to a \]
\[ g(x) = \alpha(x)f(x),\ \alpha(x) \to 0 \]
Такой вопрос: $ x \to a \;\; \overline{\overline{o}}(f(x)) + \overline{\overline{o}}(f(x)) = \overline{\overline{o}}(2f(x))?$\\
Нет, $ x \to a \;\; \overline{\overline{o}}(f(x)) + \overline{\overline{o}}(f(x)) = \overline{\overline{o}}(f(x))?$

\[ = \alpha(x)f(x) + \beta(x)f(x) = (\alpha(x) + \beta(x))f(x) \]

\[ g(x) = \underline{\underline{O}}(f(x)), x \to a \Leftrightarrow \exists U_a, \exists C > 0:\ \forall x \in U_a^{.}\;\; |g(x)| \leq C|f(x)|\]

Верно ли, что:
\[ \overline{\overline{o}}(f(x)) = \underline{\underline{O}}(f(x)) \]
\[\underline{\underline{O}}(f(x)) =\overline{\overline{o}}(f(x))  \]
Расспишем.

\[ |\overline{\overline{o}}(f(x))| = |\alpha(x)f(x)| \leq 1\cdot |f(x)|\]
Пусть $C = 1,$ тогда $\exists \delta_1 > 0: \forall x\; 0 < |x-a|<\delta_1\; |\alpha(x)| < 1$\\
Любое функция удовлетворяющая оценки o удовлетворяет O.\\
Любое функция удовлетворяющая оценки O не обязана удовлетворять о.

\section{1 замечательный предел}
\[ \frac{\sin x}{x} \xrightarrow[x\to 0]{} 1 \]

\[ \frac{\sin x}{x} = 1 + \alpha(x) \]
\[\sin x = x + \alpha(x) x = x + \overline{\overline{o}}(x)\]

\[ e^x = 1 + x +\overline{\overline{o}}(x), x\to 0 \]
\[ \cos x = 1 - \frac{x^2}{2} +\overline{\overline{o}}(x^2), x\to 0 \]
\[ (1+x)^\alpha = 1 + \alpha x +\overline{\overline{o}}(x), x\to 0 \]
\[ log(1+x) = x +\overline{\overline{o}}(x), x\to 0 \]


\[ \frac{\tg x - \sin x}{x^3} \]
\[ \tg x = \frac{\sin x}{\cos x} = \frac{x +\overline{\overline{o}}(x)}{1 - \frac{x^2}{2} + \overline{\overline{o}}(x^2)} = (x + \overline{\overline{o}}(x))(1 - \frac{x^2}{2} + \overline{\overline{o}}(x^2))^-1 = \] 
\[ = (x +\overline{\overline{o}}(x))[\ 1 + (-1)(-\frac{x^2}{2} +\overline{\overline{o}}(x^2)) + \overline{\overline{o}}(-\frac{x^2}{2} + \overline{\overline{o}}(x^2))  \ ] =(x +\overline{\overline{o}}(x)) [\ 1 + \frac{x^2}{2} + \overline{\overline{o}}(x^2) + \overline{\overline{o}}(x^2)\ ] = \]
\[(x +\overline{\overline{o}}(x)) [\ 1 + \frac{x^2}{2} + \overline{\overline{o}}(x^2)\ ]  \ = x + \frac{x^3}{2} +\overline{\overline{o}}(x^2)x + \overline{\overline{o}}(x) + \frac{x^2}{2}\overline{\overline{o}}(x) +\overline{\overline{o}}(x) +\overline{\overline{o}}(x^2) \]

\[ \frac{x^3}{2} = \frac{x^2}{2}x = \overline{\overline{o}}(x) \]
\[ \overline{\overline{o}}(x^2) = (\beta(x)x^2)x = \overline{\overline{o}}(x) \]

\end{document}
