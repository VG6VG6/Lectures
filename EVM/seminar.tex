\documentclass[12px]{article}
\usepackage{graphicx} % Required for inserting images

\usepackage[T2A]{fontenc}
\usepackage[utf8]{inputenc}
\usepackage[english, russian]{babel}
\usepackage{amsmath}

\usepackage{geometry}
 \geometry{
 a4paper,
 total={170mm,257mm},
 left=20mm,
 top=20mm,
 }

\title{ЭВМ семинары}
\author{VG6}

\usepackage{graphicx}

\makeatletter
\newcommand{\ostar}{\mathbin{\mathpalette\make@circled\star}}
\newcommand{\make@circled}[2]{%
  \ooalign{$\m@th#1\smallbigcirc{#1}$\cr\hidewidth$\m@th#1#2$\hidewidth\cr}%
}
\newcommand{\smallbigcirc}[1]{%
  \vcenter{\hbox{\scalebox{0.77778}{$\m@th#1\bigcirc$}}}%
}
\makeatother

\begin{document}

\begin{center}
	\textbf{Семинары}
\end{center}

\section{Неопределённые функции}
Доопределять надо так, чтобы получился лучший результат. Для ДНФ доопределяем единицами, чтобы результат был короче. Для КНФ - нулями.
3 этапа:
\begin{enumerate}
	\item Доопределение
	\item Минимизация
	\item Коррекция		
\end{enumerate}

Алгоритм минимизации существует в 2 вариантах и содержат оба варианта все 3 этапа.\\
Вариант 1. 
\begin{itemize}
	\item Вводится вспомогательная функция $\tilde{f}$ совпадающая с исходной на тех наборах на которых функция определена и принимающая значение 1 на запрещённых наборах.
	\item Выполняется минимизацию вспомогательной функции любым удобным способом.
	\item Строится импликантная матрица. Заголовками солбцов которой являются термы исходной функции, а заголовками строк - терми, полученные в результате минимизации вспомогательной функции. Проставляются метки, отмечающие вхождение строки в столбец и выбирается такая миимальная совокупность столбцов и строк, которая покроет всю функцию.
\end{itemize}
Вариант 2
\begin{itemize}
	\item Вводится вспомогательная функция $\tilde{f}$ которая совпадает с исх функцией на наборах, где она определена и принимает знаение 0 на запрещённых наборах.
	\item Выполняется минимизацию вспомогательной функции любым удобным способом.
	\item Строится импликантная матрица. Заголовками солбцов которой являются термы исходной функции, а заголовками строк - термы, полученные в результате минимизации вспомогательной функции. Проставляются метки, отмечающие вхождение строки в столбец и выбирается такая миимальная совокупность покрывающая все столбцы.
\end{itemize}

Условные обозначения
$ f(a, b, c) = \begin{cases}
	\sum_1 (1, 2, 3)\\
	X(4, 5)$ - запрещённые наборы$
\end{cases} $
\subsubsection*{Пример}
$ f(a, b, c, d) = \begin{cases}
	\sum_1(0, 5, 8, 12, 15), \\
	X (1, 2, 3, 10, 13, 14)
\end{cases}  \Rightarrow$
\[ 
	\tilde{f} (a, b, c, d) = \begin{cases}
	\sum_1(0, 5, 8, 12, 15, 
	1, 2, 3, 10, 13, 14)
	\end{cases} 
\]
\begin{tabular}{c c c c c c}
	& b & b & $\overline{b}$ & $\overline{b}$ & \\
	a & 1 & 1 &   & 1 & $\overline{c}$\\
	a & 1 & 1 &   & 1 & c\\
	$\overline{a}$ &   &   & 1 & 1 & c\\
	$\overline{a}$ &   & 1 & 1 & 1 & $\overline{c}$\\
	 & $\overline{d}$ & d & d & $\overline{d}$
\end{tabular}
\[ f(a, b, c, d) = ab + \overline{a}\overline{b} + \overline{b}\overline{d} + \overline{a}\overline{c}\overline{d} \]
$ \begin{tabular}{c c c c c c}
	 & 0000 & 0101 & 1000 & 1100 & 1111\\
	11-- &  & & & x & x \\
	00-- & x & & &  &  \\
	-0-0 & x &  & x &  \\
	0-01 &  & x & &  &  \\	
\end{tabular} $ 
\[ \overline{b} \overline{d} + \overline{a}\overline{c}d + ab \]

\subsubsection*{Пример (минКНФ)}
$
	f(a, b, c, d) = \begin{cases}
	$П$(3, 6, 7, 9, 11), \\
	X (0, 1, 2)
\end{cases}  \Rightarrow $ 
$ \tilde{f}(a, b, c, d) = \begin{cases}
	$П$(3, 6, 7, 9, 11, 0, 1, 2)
\end{cases}$ \\
\begin{tabular}{c c c c c c}
	   			  & b & b & $\overline{b}$ & $\overline{b}$ & \\
				a &   &   & 0 &   & $\overline{c}$\\
				a &   &   & 0 &   & c\\
	$\overline{a}$		  & 0 & 0 & 0 & 0 & c\\
	$\overline{a}$ 		  &   &   & 0 & 0 & $\overline{c}$\\
	                          & $\overline{d}$ & d & d & $\overline{d}$ \\
\end{tabular}\\
\[ (a + \overline{c}) \cdot (b + \overline{d}) \cdot (a+b) \]
\begin{tabular}{c c c c c c}
	     & 0011 & 0110 & 0111 & 1001 & 1011\\
	0-1- &  x   &  x   &  x   &      &     \\
	-0-1 &  x   &      &      &  x   &  x  \\
	00-- &  x   &      &      &      &     \\
\end{tabular}
\[ (a+\overline{c}) \cdot (b + \overline{d}) \]

\section{Выполнение операции умножения над числами с фиксированной запятой.}

Вопросик: Формат чисел с фиксированной запятой?\\
Мы будем говорить про вариант: правильная дробь. 
\subsection{Прямой код. }


\begin{tabular}{c c c c c c}
	$[x]_n$ & $Sign_x$ & $x_1$ & ... & $x_n$\\
	$[y]_n$ & $Sign_y$ & $y_1$ & ... & $y_n$
\end{tabular}
\[ |x|, |y| < 1\;\;\;\;\; [z]_n = [x]_n \cdot [y]_n = ?\]
\begin{enumerate}
	\item $Sibn_z = Sign_x \oplus Sign_y$ \\ 
		Множимое, множитель (сомножители) и произведение. 
	\item $0,.  x_1 ... x_n * 0. y_1 ... y_n $\\
		\begin{enumerate}
			\item Со старших разрядов множителя.
				\[z = x \cdot y = x(y_1\cdot 2^{-1} +y_2\cdot 2^{-2} + ... + y_n\cdot 2^{-n} ) = x\cdot 2^{-1} y_1 + x \cdot 2^{-2} y_2 + ... + x \cdot 2^{-n} y_n\]
				Обращаем внимание, что $\cdot 2^n$ - это сдвиг вправо. Всё это множимое складывается с самим собой. Всё здесь - сложение и сдвиг. 
			\item С младших разрядов множителя (Схема Горнера). 
				\[ z = ((... ((0 + xy_n) \cdot 2^{-1} + xy_{n-1})\cdot 2^{-1} + ... + xy_2)\cdot 2^{-1} +  xy_1)\cdot 2^{-1} \]
				Схема на основе накопителя (аккумулятора). 

		\end{enumerate}
\end{enumerate}

Фиксированная запятая в правильной дроби. 
\subsection{Дополнительный код.}
\begin{tabular}{c c c c c c}
	$[x]_d$ & $x_0$ & $x_1$ & ... & $x_n$\\
	$[y]_d$ & $y_0$ & $y_1$ & ... & $y_n$
\end{tabular}\\
\begin{enumerate}
	\item 
$ [z]_d = [x]_d(y_1 - y_0) + [x]_d\cdot 2^{-1}(y_2-y_1) + ... + [x]_d\cdot 2^{-n+1}(y_n-y_{n-1}) + [x]_d\cdot 2^{-n}(y_{n+1}-y_n) $
\textbf{Замечание.} Т.к. формат правильная дробь, то элемент $y_{n+1}$ мы берём за 0.\\
Значениея, которые могут появиться после вычитания:\\
\begin{tabular}{c c c}
	0 & 0 & 0\\
	0 & 1 & -1\\
	1 & 0 & 1\\
	1 & 1 & 0
\end{tabular}\\ 
	\item $[z]_d = ((...((0 + [x]_d\cdot [y_{n + 1} - y_n])\cdot 2^{-1} + [x]_d \cdot [y_n - y_{n - 1}]) \cdot 2 ^{-1} + ... + [x]_d \cdot [y_2 - y_{1}]) \cdot 2 ^{-1} + [x]_d \cdot [y_1 - y_{0}])$

\end{enumerate}

\textbf{Примерчик.}\\
\[ [x]_n = 1.1010 \]
\[ [y]_n = 0.1011 \]
\[ [z]_n - ? \]
\begin{enumerate}
	\item $Sign_z = Sign_x \oplus Sign_y = 1 \oplus 0 = 1$ 
	\item \begin{tabular}{c}
		$ 0.1010 $\\
		\hline
		$ 0.01010 \;\;\;\;\;\;\; x \cdot 2^{-1} \cdot y_1$\\
		$ 0.000000\;\;\;\;\; x \cdot 2^{-2} \cdot y_2 $\\
		$ 0.0001010\;\;\; x \cdot 2^{-3} \cdot y_3 $\\
		$ 0.00001010\; x \cdot 2^{-4} \cdot y_4 $\\
		\hline
		$0.01101110 \;\;\;\;\;\;\;\;\;\;\;\;\;\;\;\;\;$ 
	\end{tabular}
	\[ [x]_n = 1.01101110 \]
\end{enumerate}

\textbf{Ещё примерчик.}
\[ [x]_n = 1.1101 \]
\[ [y]_n = 1.1011 \]
$ [z]_n$ - с Младших разрядов. 
\begin{enumerate}
	\item $Sign_z = 0$
	\item \begin{tabular}{c}	
		$0.0000$\\
		$0.1101$   $0 + x y_4 = \sum_1$\\
		\hline 
		$0.01101 \cdot \sum_1 \cdot 2^{-1}$\\
		$0.1101$   $x\cdot y_3$ \\
		\hline
		$1.00111$   $\sum_1 \cdot 2^{-3} - xy_2 = \sum_2$\\
		0.100111   $\sum_2 \cdot 2^{-1}$\\
		0.0000    $x \cdot y_2$\\
		\hline 
		0.100111   $\sum_3$\\
		0.0100111  $\sum_3 \cdot 2^{1}$\\
		0.1101\\
		\hline 
		1.0001111  $\cdot 2^{-1}$
		$[z]_n = 0.10001111$  $\sum_n$
	\end{tabular}
\end{enumerate}

\textbf{И ещё.}\\
\[ [x]_g = 0.10101 \]
\[ [y]_g = 1.01101 \]
$ [z]_g$ - со Старших разрядов. \\
\begin{tabular}{c | c | c| c | c}
	i & $u_i-y_{i-1}$ & () & & \\
	\hline
	i=1 & 0 - 1 & -1 & 1.01011 & $[y_1-y_0] \cdot 2^{-0} [x]_g$\\
	i=2 & 1 - 0 & 1 &  0.010101 & $[y_2-y_1] \cdot 2^{-1} [x]_g$\\
	i=3 & 1 - 1 & 0 &  0.0000000 & $[y_3-y_2] \cdot 2^{-2} [x]_g$\\
\end{tabular}












\end{document}
