\documentclass[12px]{article}
\usepackage{graphicx} % Required for inserting images

\usepackage[T2A]{fontenc}
\usepackage[utf8]{inputenc}
\usepackage[english, russian]{babel}
\usepackage{amsmath}

\usepackage{geometry}
 \geometry{
 a4paper,
 total={170mm,257mm},
 left=20mm,
 top=20mm,
 }

\title{ЭВМ семинары}
\author{VG6}

\begin{document}

\begin{center}
	\textbf{Семинары}
\end{center}

\section{Неопределённые функции}
Доопределять надо так, чтобы получился лучший результат. Для ДНФ доопределяем единицами, чтобы результат был короче. Для КНФ - нулями.
3 этапа:
\begin{enumerate}
	\item Доопределение
	\item Минимизация
	\item Коррекция		
\end{enumerate}

Алгоритм минимизации существует в 2 вариантах и содержат оба варианта все 3 этапа.\\
Вариант 1. 
\begin{itemize}
	\item Вводитяс вспомогательная функция \tilde{f} совпадающая с исходной на тех наборах на которых функция определена и принимающая значение 1 на запрещённых наборах.
	\item Выполняется минимизацию вспомогательной функции любым удобным способом.
	\item Строится импликантная матрица. Заголовками солбцов которой являются термы исходной функции, а заголовками строк - терми, полученные в результате минимизации вспомогательной функции. Проставляются метки, отмечающие вхождение строки в столбец и выбирается такая миимальная совокупность столбцов и строк, которая покроет всю функцию.
\end{itemize}
Вариант 2
\begin{itemize}
	\item Вводится вспомогательная функция $\tilde{f}$ которая совпадает с исх функцией на наборах, где она определена и принимает знаение 0 на запрещённых наборах.
	\item Выполняется минимизацию вспомогательной функции любым удобным способом.
	\item Строится импликантная матрица. Заголовками солбцов которой являются термы исходной функции, а заголовками строк - термы, полученные в результате минимизации вспомогательной функции. Проставляются метки, отмечающие вхождение строки в столбец и выбирается такая миимальная совокупность покрывающая все столбцы.
\end{itemize}

Условные обозначения
$ f(a, b, c) = \begin{cases}
	\sum_1 (1, 2, 3)\\
	X(4, 5)$ - запрещённые наборы$
\end{cases} $
\subsubsection*{Пример}
$ f(a, b, c, d) = \begin{cases}
	\sum_1(0, 5, 8, 12, 15), \\
	X (1, 2, 3, 10, 13, 14)
\end{cases}  \Rightarrow$
$\tilde{f} (a, b, c, d) = \begin{cases}
	\sum_1(0, 5, 8, 12, 15, 
	1, 2, 3, 10, 13, 14)
\end{cases} $\\

\begin{tabular}{c c c c c c}
	& b & b & \overline{b} & \overline{b} & \\
	a & 1 & 1 &   & 1 & \overline{c}\\
	a & 1 & 1 &   & 1 & c\\
	\overline{a} &   &   & 1 & 1 & c\\
	\overline{a} &   & 1 & 1 & 1 & \overline{c}\\
	 & \overline{d} & d & d & \overline{d}
\end{tabular}\\
\[ f(a, b, c, d) = ab + \overline{a}\overline{b} + \overline{b}\overline{d} + \overline{a}\overline{c}\overline{d} \]
\begin{tabular}{c c c c c c}
	 & 0000 & 0101 & 1000 & 1100 & 1111\\
	11-- &  & & & x & x \\
	00-- & x & & &  &  \\
	-0-0 & x &  & x &  \\
	0-01 &  & x & &  &  \\	
\end{tabular}
\[ bd + \overline{a}\overline{c}d + ab \]

\subsubsection*{Пример (минКНФ)}
$ f(a, b, c, d) = \begin{cases}
	$П$(3, 6, 7, 9, 11), \\
	X (0, 1, 2)
\end{cases}  \Rightarrow$
$ \tilde{f}(a, b, c, d) = \begin{cases}
	$П$(3, 6, 7, 9, 11, 0, 1, 2)
\end{cases} \\
\begin{tabular}{c c c c c c}
	   			  & b & b & \overline{b} & \overline{b} & \\
				a &   &   & 0 &   & \overline{c}\\
				a &   &   & 0 &   & c\\
	\overline{a}		  & 0 & 0 & 0 & 0 & c\\
	\overline{a} 		  &   &   & 0 & 0 & \overline{c}\\
	                          & \overline{d} & d & d & \overline{d}\\
\end{tabular}\\

\[ (a + \overline{c}) \cdot (b + \overline{d}) \cdot ab \]
\begin{tabular}{c c c c c c}
	     & 0011 & 0110 & 0111 & 1001 & 1011\\
	0-1- &  x   &  x   &  x   &      &     \\
	-0-1 &  x   &      &      &  x   &  x  \\
	00-- &  x   &      &      &      &     \\
\end{tabular}
\[ (a+\overline{c}) \cdot (b + \overline{d}) \]

\end{document}
