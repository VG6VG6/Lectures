\documentclass[12pt, paper]{article}

\usepackage{graphicx}
\usepackage{amsmath, amssymb}
\usepackage[T2A]{fontenc}
\usepackage[english, russian]{babel}
\usepackage[utf8]{inputenc}
\usepackage[margin=2cm]{geometry}
\usepackage[most]{tcolorbox}
\usepackage{caption}
\usepackage{enumitem}

\tcbuselibrary{breakable}
\tcbset{
  width=0.9\textwidth,
  halign=justify,
  center,
  breakable,
  colback=white
}

\title{Конспект по математическому анализу}
\author{Голубов Владислав}
\date{Сентябрь 2025}

\newcommand{\N}{\mathbb{N}}
\newcommand{\Q}{\mathbb{Q}}
\newcommand{\Z}{\mathbb{Z}}
\newcommand{\R}{\mathbb{R}}
\newcommand{\eps}{\varepsilon}

\begin{document}
\maketitle
\tableofcontents
\newpage

\[ Ax + by + Cz + D = 0 \]
\[  \]
Я всё проебал, надо написать хз что это.\\
Определение. Отклонение точки от плоскости
\[ M_0 (x_0, _0, z_0) \]
\[ \delta (M_0, \pi) = x_0 \cos \alpha + y_0 \cos \beta z_0 \cos \gamma \]

\section{ \S 8. Прямая в пространстве. }
OXYZ - ДПСК\\
\textbf{Определение.} Направленный вектор L.
\[\overline{p} \not= 0,\;\ \overline{q} || L\]
\[ M_0 = (x_0, y_0, z_0) \]
\[
	\overline{q} = {l, m, n}\]
\[
	M = (x, y, z)\]
	\[ M \in L \Leftrightarrow \overline{M_0M} || \overline{q} \]
	\begin{enumerate}
		\item \[ \frac{x-x_0}{l} = \frac{y - y_0}{m} = \frac{z-z_0}{n} \]
	Канонические уравнения L.
		\item \[ \overline{M_0M} = t \overline{q} \]
			$
			\begin{cases}
				x = x_0 + lt, t \in (-\infty , +\infty)\\
				y = y_0 + mt\\
				z = z_0 + nt
			\end{cases}
			$
			$
		\item \begin{cases}
			П1: $A_1x + B_1y + C_1z + D_1 = 0$\\
			П2: $ A_2x + B_2y + C_2z + D_2 = 0 $
\end{cases}
$
\[\overline{q} = [\overline{N_1}, \overline{N_2}] \]
	\end{enumerate}
 
	Определение. Угол между прямой и плоскость - это угол между прямой и её проекцией на эту плоскость.
	\[ \overline{q} = {l, m, n} \]
	\[ \overline{N} = {A, B, C} \]
	\[ \cos \psi = \sin \phi = \frac{Al + Bm + Cn}{\sqrt{A^2 + B^2 + C^2} \cdot \sqrt{l^2 + m^2 + n ^2}} \]
\section{Алгебраические линии и кривые 2 порядка}
\subsection{Элипс. Вывод канонического уравнения. Свойства элипса.}
\textbf{Элипс} - множество точек плоскости таких, что сумма расстояний от них до фиксированных точек той же плоскость постоянна и равна 2a. $ r_1 + r_2 = 2a $\\
\textbf{Фиксированные точки F1, F2} - фокусы.\\
\textbf{Длинны r1, r2} - факальные радиусы.\\
\textbf{Расстояние между F1, F2} - фокусное расстояние = 2c. с может быть равен 0, тогда будет окружность.\\
Окружность частный случай элипса с фокусным растоянием 0.\\
По неравенству треугольника: $ a > c $

\subsection{Вывод уравнения элипса}
Пусть фокусное расстояние не равно 0.\\
!!! INSERT IMAGE.\\
\begin{equation}
	r_1 + r_2 = 2a
	\label{eq:Вывод}
\end{equation}
\begin{equation}	
	\sqrt{ (x + c)^2 + y^2 } + \sqrt{(x - c)^2 + y^2}
	\label{eq:}
\end{equation}
\[	x^2 + 2cy + c^2 + y^2 + x^2 - 2cx + c^2 + y^2  + 2\sqrt{(x^2 + 2cx + c^2 + y^2)(x^2 - 2cx + c^2 + y^2)} \]
\[ 2( x^2 + y^2 + c^2) + \sqrt{(x^2 + y^2 + c^2) ^ 2 - 4c^2x^2} = 4a^2 \]
\[ x^2 + y^2 + c^2 = t^2 \]
\[ \sqrt{t^4 - 4c^2x^2} = 2a^2 - t^2 \Rightarrow t^4 - 4c^2x^2 = 4a^4 - 4a^2t^2 + t^4 - c^2x^2 = a^4 - a^2(x^2 + y^2 + c^2) \]
\[ x^2(a^2 - c^2) + a^2y^2 = a^2(a^2(a^2-c^2) = \]
\[ (a^2 - c^2) > 0 \]
\[ = \underline{a^2 - c^2 = b^2}\;\;\;\;\;\;\; \underline{a  \geq b} \]
\[ b^2x^2 + a^2y^2 = a^2b^2 \]
\begin{equation}
	\frac{x^2}{a^2} + \frac{y^2}{b^2} = 1
	\label{eq:}
\end{equation}
$(3) \Rightarrow (2) - ?$\\
\[ y^2 = (1 - \frac{x^2}{a^2})b^2 \]
\[ r_1 = \sqrt{(x+c)^2 + y^2} = \sqrt{x^2 + 2cx + c^2 + b^2 - \frac{b^2}{a^2}x^2} = \sqrt{x^2 + 2cx + c^2 + a^2 - c^2 - \frac{a^2 - c^2}{a^2}x^2} =\]
\[\sqrt{a^2 + 2cx + (\frac{c}{a}x)^2} = \sqrt{(a + \frac{c}{a}x)^2} = |a + \frac{c}{a}x| \]
\[ r_1 = |a + \frac{c}{a}x| \;\;\;\;\;\; r_2 = \sqrt{(x - c)^2 + y^2} \]
\[ r_2 = |a - \frac{c}{a}x| \]
\begin{enumerate}
	\item $r_1 = |a + \frac{c}{a}x| = a + \frac{c}{a}x$\\
		$|x| \leq a,\ c < a$
	\item $r_2 = |a - \frac{c}{a}x| = a - \frac{c}{a}x$
\end{enumerate}
Т.о. мы доказали равносильность преобразования и вывода формулы (3).\\
a - большая полуось\\
b - малая полуось\\
Прямоугольник со сторонами a, b - основной прямоугольник для элипса.

\begin{equation}
	\begin{cases}
		r_2 = a + \frac{c}{a}x\\
		r_2 = a - \frac{c}{a}x
	\end{cases}
	\label{eq:}
\end{equation}
Гипербола - Множество точек плоскости таких, что модуль разности расстояний до двух фиксированных точек плоскости величина постоянная и равная 2a.\\
\textbf{Фиксированные точки F1, F2} - фокусы.\\
\textbf{Длинны r1, r2} - факальные радиусы.\\
\textbf{Расстояние между F1, F2} - фокусное расстояние = 2c.
\[c > a \;\;\;\;\; c > 0\]
Вводим каноническую систему координат.\\
\[ \frac{x^2}{a^2} - \frac{y^2}{b^2} = 1 \]


\end{document}
